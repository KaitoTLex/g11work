\documentclass{article}
\usepackage{setspace}
\usepackage{hyperref}
\usepackage[citestyle=authoryear,bibstyle=mla,natbib=true]{biblatex}
\usepackage[margin=25mm]{geometry}
\addbibresource{works.bib}
\title{Women in the Antebellum period:}
\author{Warren Lin\\ San Ramon Valley High School \\ Document composed in \LaTeX{}}
\begin{document}
\maketitle
%\newpage
%Write the abstract last
\begin{abstract}


\end{abstract}

\section{Introduction}
During the antebellum period Female groups reformed. Women with
\subsection{claim}
Selective groups of womens used their political influence, economic freedom and cultural reform to notion women's ability in American Society for better or for worse.


%Point 1: #NOT TRUMP - Challanging the Political Standards as civil servants
%How did they use their influence in supporting the Seneca Falls Convention, Expressed Education and Sponsored Abolitionist movement  
\section{Non-Political Politics: Reforms by female Civil Servants.}

%Point 2: Econ: Worker's Rights, Brothel Workers.
\section{The Conditions of Women Workers.}
Some women recieved Slight economic freedom from low-paying occupations such as textile mills, domestic servitude, ad harlotry; Many women during this time period endured these hellish enviorments for a marginal change in the economic systems of rights, others fantasized about the beauty of republican Motherhood. Worker's conditions for women were poor, textile mills proved to have a high fatality rate, which encouraged some women to band together and change their conditions. However, Women lacked the Economic freedom of creating laws or unionizing. An example of this could be seen with the Lowell girls and the First Female union that reformed the conditions dispite having little economic freedom. Women working in brothels were subject to aggressive clients and could contract diseases. Working as prostitutes engaging in harlotry proved to both have lack of financial compensation as well as safety. Women working in these conditions barely had enough to cover expenses for themselves; due to the commonality of unprotected sex, prostitutes would have to nourish a child. These added expenses prompted women continue working under these poor conditions, relief recieved for prostitutes which provided a significant leap towards better conditions for these selective groups of women.
%Women received numerous economic advances during the antebellum era. The slight economic freedom from low-paying occupations like textile mills, domestic slavery, and harlotry made single women vulnerable to manipulation. The textile mills usually paid poorly, hardly making up for the dangerous working conditions. The Lowell girls were the first to band together and defend their own safety rights. On the other hand, prostitution was a popular means for women to earn a meager living wage that would cover their expenses in cities. They were the first to successfully form a union in order to demand higher pay and a safer workplace. The lowell girls banded together on strike demanding their overlords to increase wages and improve safety within the factories. Furthermore, women frequently turned to prostitution as a means of earning a meager living wage to cover their expenses in cities. There is a high fatality rate in the industry as a result of the prevalence of aggressive murder brought on by customers' drunkenness and sexually transmitted diseases from frequent sexual contact. Many unmarried women were employed as domestic servants and were "often exploited by the master,"\parencite{henretta} which led to their departure. However, as they had nowhere else to go, these women returned to the Harlotry. The issues faced by married women were distinct: Rights to Property. These women would frequently have diminished ownership rights after marriage because ownership is nullified upon marriage. Some males take advantage of this, gaining authority over a woman's property when they marry her. Finding a place where the money made from these factories could be used with a return on income would also be difficult for many lower-class single women. All things considered, these fundamental issues played a part in the few marginal advances that women made prior to the rise of labor unions and the reform of society without formal political rights.

\subsection{Lowell Girls}
Usually, the Lowell Girls were unmarried because they missed family or had poor family finances. Many strive to marriage as their conditions imporve from this hellish textile mill to a mother caring for 2 children. Due to the design of factories and shifts, women were  women would have to put up with these difficult conditions, and their pay would allow them to purchase items. Children and women would put in endless hours for a meager profit. Usually, these meager marginal advantages were insufficient to make up for the hazardous working conditions. For instance, ladies would normally cut the sticks in match manufacturers, while kids would dip in red phosphorous.\citep[117]{stansell1982}. A mill in \emph{Lowell, Massachusetts} were the first to unionize and fight for the rights and wages of these workers. A mill in \emph{Lowell, Massachusetts} were the first to unionize and fight for the rights and wages of these workers. The Lowell Mill Girls banded together, due to the expression orginating from \emph{Lowell Offering}, the Lowell Mill Girls were able to draw the conclusion that they were trapped in a cycle of consumerism and production. The \emph{Lowell Offering} would commonily have discourse regarding republican daughter-/motherhood which commonily discussed jobs and ideas beyond the horrid factory jobs and lowly wages. \parencite{kanzler2005} This contributed to the desire for greater conditions and more access to litrature such as the \emph{Lowell Offering}. This challange of freedom for young women regarding tradational gender roles of republican motherhood to the society of Early United States. Other mill girls followed the virtues of \emph{The Cult of Domesticity} which valued Purity over wealth; this provided a belief for some Lowell Girls which could link back to the \emph{Lowell Offering} on commodity items.

%\section{Economic: Property and Workers}
%Women received numerous economic advances during the antebellum era. The slight economic freedom from low-paying occupations like textile mills, domestic slavery, and harlotry made single women vulnerable to manipulation. The textile mills usually paid poorly, hardly making up for the dangerous working conditions. The Lowell girls were the first to band together and defend their own safety rights. On the other hand, prostitution was a popular means for women to earn a meager living wage that would cover their expenses in cities. They were the first to successfully form a union in order to demand higher pay and a safer workplace. The lowell girls banded together on strike demanding their overlords to increase wages and improve safety within the factories. Furthermore, women frequently turned to prostitution as a means of earning a meager living wage to cover their expenses in cities. There is a high fatality rate in the industry as a result of the prevalence of aggressive murder brought on by customers' drunkenness and sexually transmitted diseases from frequent sexual contact. Many unmarried women were employed as domestic servants and were "often exploited by the master,"\parencite{henretta} which led to their departure. However, as they had nowhere else to go, these women returned to the Harlotry. The issues faced by married women were distinct: Rights to Property. These women would frequently have diminished ownership rights after marriage because ownership is nullified upon marriage. Some males take advantage of this, gaining authority over a woman's property when they marry her. Finding a place where the money made from these factories could be used with a return on income would also be difficult for many lower-class single women. All things considered, these fundamental issues played a part in the few marginal advances that women made prior to the rise of labor unions and the reform of society without formal political rights.
\subsection{Brothel Workers and Domestic Servants}
Due to urbanization, young women seeking economic liberties lived in cities in which they served as domestic servants. Women who were typically working as domestic servants were exploited by their masters sexually. Although some hired Western European immigrants and provided for them, \parencite{west1992} the treatment of domestic servants was typically poor. Many aristocrats hire the persons with the best physical appearances. This is because these aristocrats disregard the laws and religious ethics and rather engage in sexual activities with their domestic servants. Many servants birthed children in this process, which proved poor working conditions. Because of this sexual exploitation, many women turned to sex as a form of income. Although prostitution was illegal in New York and other states, it was loosely enforced, which did not deter brothels and prostitution. Sexually Transmitted Diseases/Infections were common with these women who sold their bodies. The abundance of these men seeking to engage in sexual intercourse while drunk could lead to violent outbursts in which women were typically killed. This can be seen with the case of \emph{Helen Jewett}, in which her head was gashed and fire was set to her bed. The police concluded the case after 2 hours, and this news became nationwide. This is linked to the influx of teenage boys in the city leading to a higher demand for prostitutes. But because of the lack of development in the teenage mind, outbursts of anger due to loyalty can be witnessed with this case. \parencite{patricia1836} Although this applied to a selective group of women, they were stuck in a unbounded loop of exploitation and many could not leave their occupations especially brothelsThis prompted many well-off female patrons to start up asylums to protect these young adults from further attacks and sexual exploitation. \parencite{lee2006} The suffering of women prompted a radical change that provided a safe haven for prostitutes who are always living on the edge of death from "pleasure." This unhealthy obsession was how many women depended on it as a viable source of income, despite the harshness and dangers of this line of work. Because of this, many wealthy women thought that it was correct to build asylums to benefit their own people. This significantly increased independence in the industry, as there is now a safe place in which women would not be sexually exploited and horribly murdered.


%Point 3: Shake it off 
%Cultural Influences: Conservative Gender roles(Republican Motherhood), Temprance Movement and Spiritual Society(Shakers).
\section{Culturally Challanging: Sex\&Gender, Ethics, and Religion }

\newpage
\printbibliography
\end{document}
