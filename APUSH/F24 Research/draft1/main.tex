\documentclass{article}
\usepackage{setspace}
\usepackage{hyperref}
\usepackage[citestyle=authoryear,bibstyle=mla,natbib=true]{biblatex}
\addbibresource{works.bib}
\title{Women in the Antebellum period: Taking Advantage of the System}
\author{Warren Lin\\ San Ramon Valley High School \\ Document composed in \LaTeX{}}
\begin{document}
\maketitle
\newpage

%Write the abstract last
\begin{abstract}


\end{abstract}

\newpage

\doublespacing
\tableofcontents
\singlespacing

\newpage

\section{Introduction}
\subsection{Background}
\subsection{claim}

%Point 1: #NOT TRUMP
\section{Non-Political Politics}
\subsection{Seneca Falls Convention}
\subsection{Educational Expressionism}
\subsection{Abolitionist Movement}

%Point 2: Econ
\section{Economic: Property and Workers}
Women received numerous economic advances during the antebellum era. The slight economic freedom from low-paying occupations like textile mills, domestic slavery, and harlotry made single women vulnerable to manipulation. The textile mills usually paid poorly, hardly making up for the dangerous working conditions. The Lowell girls were the first to band together and defend their own safety rights. On the other hand, prostitution was a popular means for women to earn a meager living wage that would cover their expenses in cities. They were the first to successfully form a union in order to demand higher pay and a safer workplace. The lowell girls banded together on strike demanding their overlords to increase wages and improve safety within the factories. Furthermore, women frequently turned to prostitution as a means of earning a meager living wage to cover their expenses in cities. There is a high fatality rate in the industry as a result of the prevalence of aggressive murder brought on by customers' drunkenness and sexually transmitted diseases from frequent sexual contact. Many unmarried women were employed as domestic servants and were "often exploited by the master,"\parencite{henretta} which led to their departure. However, as they had nowhere else to go, these women returned to the Harlotry. The issues faced by married women were distinct: Rights to Property. These women would frequently have diminished ownership rights after marriage because ownership is nullified upon marriage. Some males take advantage of this, gaining authority over a woman's property when they marry her. Finding a place where the money made from these factories could be used with a return on income would also be difficult for many lower-class single women. All things considered, these fundamental issues played a part in the few marginal advances that women made prior to the rise of labor unions and the reform of society without formal political rights.
\subsection{Lowell Girls}
Despite having less financial freedom, women were harder workers than men dispite having to labor in difficult conditions. Usually, these women were unmarried because they missed family or had poor family finances. Due to the design of factories and shifts, women were  women would have to put up with these difficult conditions, and their pay would allow them to purchase items. Children and women would put in endless hours for a meager profit. Usually, these meager marginal advantages were insufficient to make up for the hazardous working conditions. For instance, ladies would normally cut the sticks in match manufacturers, while kids would dip in red phosphorous.\citep[117]{stansell1982}. A mill in \emph{Lowell, Massachusetts} were the first to unionize and fight for the rights and wages of these workers. The Lowell Mill Girls banded together anuiid
\subsection{Brothel Workers and Domestic Servants}
\subsection{Property Rights}

%Point 3: Shake it off
\section{Cultural Influences}
\subsection{Challanging conservative gender roles}
\subsection{Temprance Society}
\subsection{Spiritual Socialism}

\newpage
\printbibliography
\end{document}
